
% Default to the notebook output style

    


% Inherit from the specified cell style.




    
\documentclass[11pt]{article}

    
    
    \usepackage[T1]{fontenc}
    % Nicer default font (+ math font) than Computer Modern for most use cases
    \usepackage{mathpazo}

    % Basic figure setup, for now with no caption control since it's done
    % automatically by Pandoc (which extracts ![](path) syntax from Markdown).
    \usepackage{graphicx}
    % We will generate all images so they have a width \maxwidth. This means
    % that they will get their normal width if they fit onto the page, but
    % are scaled down if they would overflow the margins.
    \makeatletter
    \def\maxwidth{\ifdim\Gin@nat@width>\linewidth\linewidth
    \else\Gin@nat@width\fi}
    \makeatother
    \let\Oldincludegraphics\includegraphics
    % Set max figure width to be 80% of text width, for now hardcoded.
    \renewcommand{\includegraphics}[1]{\Oldincludegraphics[width=.8\maxwidth]{#1}}
    % Ensure that by default, figures have no caption (until we provide a
    % proper Figure object with a Caption API and a way to capture that
    % in the conversion process - todo).
    \usepackage{caption}
    \DeclareCaptionLabelFormat{nolabel}{}
    \captionsetup{labelformat=nolabel}

    \usepackage{adjustbox} % Used to constrain images to a maximum size 
    \usepackage{xcolor} % Allow colors to be defined
    \usepackage{enumerate} % Needed for markdown enumerations to work
    \usepackage{geometry} % Used to adjust the document margins
    \usepackage{amsmath} % Equations
    \usepackage{amssymb} % Equations
    \usepackage{textcomp} % defines textquotesingle
    % Hack from http://tex.stackexchange.com/a/47451/13684:
    \AtBeginDocument{%
        \def\PYZsq{\textquotesingle}% Upright quotes in Pygmentized code
    }
    \usepackage{upquote} % Upright quotes for verbatim code
    \usepackage{eurosym} % defines \euro
    \usepackage[mathletters]{ucs} % Extended unicode (utf-8) support
    \usepackage[utf8x]{inputenc} % Allow utf-8 characters in the tex document
    \usepackage{fancyvrb} % verbatim replacement that allows latex
    \usepackage{grffile} % extends the file name processing of package graphics 
                         % to support a larger range 
    % The hyperref package gives us a pdf with properly built
    % internal navigation ('pdf bookmarks' for the table of contents,
    % internal cross-reference links, web links for URLs, etc.)
    \usepackage{hyperref}
    \usepackage{longtable} % longtable support required by pandoc >1.10
    \usepackage{booktabs}  % table support for pandoc > 1.12.2
    \usepackage[inline]{enumitem} % IRkernel/repr support (it uses the enumerate* environment)
    \usepackage[normalem]{ulem} % ulem is needed to support strikethroughs (\sout)
                                % normalem makes italics be italics, not underlines
    

    
    
    % Colors for the hyperref package
    \definecolor{urlcolor}{rgb}{0,.145,.698}
    \definecolor{linkcolor}{rgb}{.71,0.21,0.01}
    \definecolor{citecolor}{rgb}{.12,.54,.11}

    % ANSI colors
    \definecolor{ansi-black}{HTML}{3E424D}
    \definecolor{ansi-black-intense}{HTML}{282C36}
    \definecolor{ansi-red}{HTML}{E75C58}
    \definecolor{ansi-red-intense}{HTML}{B22B31}
    \definecolor{ansi-green}{HTML}{00A250}
    \definecolor{ansi-green-intense}{HTML}{007427}
    \definecolor{ansi-yellow}{HTML}{DDB62B}
    \definecolor{ansi-yellow-intense}{HTML}{B27D12}
    \definecolor{ansi-blue}{HTML}{208FFB}
    \definecolor{ansi-blue-intense}{HTML}{0065CA}
    \definecolor{ansi-magenta}{HTML}{D160C4}
    \definecolor{ansi-magenta-intense}{HTML}{A03196}
    \definecolor{ansi-cyan}{HTML}{60C6C8}
    \definecolor{ansi-cyan-intense}{HTML}{258F8F}
    \definecolor{ansi-white}{HTML}{C5C1B4}
    \definecolor{ansi-white-intense}{HTML}{A1A6B2}

    % commands and environments needed by pandoc snippets
    % extracted from the output of `pandoc -s`
    \providecommand{\tightlist}{%
      \setlength{\itemsep}{0pt}\setlength{\parskip}{0pt}}
    \DefineVerbatimEnvironment{Highlighting}{Verbatim}{commandchars=\\\{\}}
    % Add ',fontsize=\small' for more characters per line
    \newenvironment{Shaded}{}{}
    \newcommand{\KeywordTok}[1]{\textcolor[rgb]{0.00,0.44,0.13}{\textbf{{#1}}}}
    \newcommand{\DataTypeTok}[1]{\textcolor[rgb]{0.56,0.13,0.00}{{#1}}}
    \newcommand{\DecValTok}[1]{\textcolor[rgb]{0.25,0.63,0.44}{{#1}}}
    \newcommand{\BaseNTok}[1]{\textcolor[rgb]{0.25,0.63,0.44}{{#1}}}
    \newcommand{\FloatTok}[1]{\textcolor[rgb]{0.25,0.63,0.44}{{#1}}}
    \newcommand{\CharTok}[1]{\textcolor[rgb]{0.25,0.44,0.63}{{#1}}}
    \newcommand{\StringTok}[1]{\textcolor[rgb]{0.25,0.44,0.63}{{#1}}}
    \newcommand{\CommentTok}[1]{\textcolor[rgb]{0.38,0.63,0.69}{\textit{{#1}}}}
    \newcommand{\OtherTok}[1]{\textcolor[rgb]{0.00,0.44,0.13}{{#1}}}
    \newcommand{\AlertTok}[1]{\textcolor[rgb]{1.00,0.00,0.00}{\textbf{{#1}}}}
    \newcommand{\FunctionTok}[1]{\textcolor[rgb]{0.02,0.16,0.49}{{#1}}}
    \newcommand{\RegionMarkerTok}[1]{{#1}}
    \newcommand{\ErrorTok}[1]{\textcolor[rgb]{1.00,0.00,0.00}{\textbf{{#1}}}}
    \newcommand{\NormalTok}[1]{{#1}}
    
    % Additional commands for more recent versions of Pandoc
    \newcommand{\ConstantTok}[1]{\textcolor[rgb]{0.53,0.00,0.00}{{#1}}}
    \newcommand{\SpecialCharTok}[1]{\textcolor[rgb]{0.25,0.44,0.63}{{#1}}}
    \newcommand{\VerbatimStringTok}[1]{\textcolor[rgb]{0.25,0.44,0.63}{{#1}}}
    \newcommand{\SpecialStringTok}[1]{\textcolor[rgb]{0.73,0.40,0.53}{{#1}}}
    \newcommand{\ImportTok}[1]{{#1}}
    \newcommand{\DocumentationTok}[1]{\textcolor[rgb]{0.73,0.13,0.13}{\textit{{#1}}}}
    \newcommand{\AnnotationTok}[1]{\textcolor[rgb]{0.38,0.63,0.69}{\textbf{\textit{{#1}}}}}
    \newcommand{\CommentVarTok}[1]{\textcolor[rgb]{0.38,0.63,0.69}{\textbf{\textit{{#1}}}}}
    \newcommand{\VariableTok}[1]{\textcolor[rgb]{0.10,0.09,0.49}{{#1}}}
    \newcommand{\ControlFlowTok}[1]{\textcolor[rgb]{0.00,0.44,0.13}{\textbf{{#1}}}}
    \newcommand{\OperatorTok}[1]{\textcolor[rgb]{0.40,0.40,0.40}{{#1}}}
    \newcommand{\BuiltInTok}[1]{{#1}}
    \newcommand{\ExtensionTok}[1]{{#1}}
    \newcommand{\PreprocessorTok}[1]{\textcolor[rgb]{0.74,0.48,0.00}{{#1}}}
    \newcommand{\AttributeTok}[1]{\textcolor[rgb]{0.49,0.56,0.16}{{#1}}}
    \newcommand{\InformationTok}[1]{\textcolor[rgb]{0.38,0.63,0.69}{\textbf{\textit{{#1}}}}}
    \newcommand{\WarningTok}[1]{\textcolor[rgb]{0.38,0.63,0.69}{\textbf{\textit{{#1}}}}}
    
    
    % Define a nice break command that doesn't care if a line doesn't already
    % exist.
    \def\br{\hspace*{\fill} \\* }
    % Math Jax compatability definitions
    \def\gt{>}
    \def\lt{<}
    % Document parameters
    \title{Basics of Programming I -- Python Workshops}
    
    
    

    % Pygments definitions
    
\makeatletter
\def\PY@reset{\let\PY@it=\relax \let\PY@bf=\relax%
    \let\PY@ul=\relax \let\PY@tc=\relax%
    \let\PY@bc=\relax \let\PY@ff=\relax}
\def\PY@tok#1{\csname PY@tok@#1\endcsname}
\def\PY@toks#1+{\ifx\relax#1\empty\else%
    \PY@tok{#1}\expandafter\PY@toks\fi}
\def\PY@do#1{\PY@bc{\PY@tc{\PY@ul{%
    \PY@it{\PY@bf{\PY@ff{#1}}}}}}}
\def\PY#1#2{\PY@reset\PY@toks#1+\relax+\PY@do{#2}}

\expandafter\def\csname PY@tok@w\endcsname{\def\PY@tc##1{\textcolor[rgb]{0.73,0.73,0.73}{##1}}}
\expandafter\def\csname PY@tok@c\endcsname{\let\PY@it=\textit\def\PY@tc##1{\textcolor[rgb]{0.25,0.50,0.50}{##1}}}
\expandafter\def\csname PY@tok@cp\endcsname{\def\PY@tc##1{\textcolor[rgb]{0.74,0.48,0.00}{##1}}}
\expandafter\def\csname PY@tok@k\endcsname{\let\PY@bf=\textbf\def\PY@tc##1{\textcolor[rgb]{0.00,0.50,0.00}{##1}}}
\expandafter\def\csname PY@tok@kp\endcsname{\def\PY@tc##1{\textcolor[rgb]{0.00,0.50,0.00}{##1}}}
\expandafter\def\csname PY@tok@kt\endcsname{\def\PY@tc##1{\textcolor[rgb]{0.69,0.00,0.25}{##1}}}
\expandafter\def\csname PY@tok@o\endcsname{\def\PY@tc##1{\textcolor[rgb]{0.40,0.40,0.40}{##1}}}
\expandafter\def\csname PY@tok@ow\endcsname{\let\PY@bf=\textbf\def\PY@tc##1{\textcolor[rgb]{0.67,0.13,1.00}{##1}}}
\expandafter\def\csname PY@tok@nb\endcsname{\def\PY@tc##1{\textcolor[rgb]{0.00,0.50,0.00}{##1}}}
\expandafter\def\csname PY@tok@nf\endcsname{\def\PY@tc##1{\textcolor[rgb]{0.00,0.00,1.00}{##1}}}
\expandafter\def\csname PY@tok@nc\endcsname{\let\PY@bf=\textbf\def\PY@tc##1{\textcolor[rgb]{0.00,0.00,1.00}{##1}}}
\expandafter\def\csname PY@tok@nn\endcsname{\let\PY@bf=\textbf\def\PY@tc##1{\textcolor[rgb]{0.00,0.00,1.00}{##1}}}
\expandafter\def\csname PY@tok@ne\endcsname{\let\PY@bf=\textbf\def\PY@tc##1{\textcolor[rgb]{0.82,0.25,0.23}{##1}}}
\expandafter\def\csname PY@tok@nv\endcsname{\def\PY@tc##1{\textcolor[rgb]{0.10,0.09,0.49}{##1}}}
\expandafter\def\csname PY@tok@no\endcsname{\def\PY@tc##1{\textcolor[rgb]{0.53,0.00,0.00}{##1}}}
\expandafter\def\csname PY@tok@nl\endcsname{\def\PY@tc##1{\textcolor[rgb]{0.63,0.63,0.00}{##1}}}
\expandafter\def\csname PY@tok@ni\endcsname{\let\PY@bf=\textbf\def\PY@tc##1{\textcolor[rgb]{0.60,0.60,0.60}{##1}}}
\expandafter\def\csname PY@tok@na\endcsname{\def\PY@tc##1{\textcolor[rgb]{0.49,0.56,0.16}{##1}}}
\expandafter\def\csname PY@tok@nt\endcsname{\let\PY@bf=\textbf\def\PY@tc##1{\textcolor[rgb]{0.00,0.50,0.00}{##1}}}
\expandafter\def\csname PY@tok@nd\endcsname{\def\PY@tc##1{\textcolor[rgb]{0.67,0.13,1.00}{##1}}}
\expandafter\def\csname PY@tok@s\endcsname{\def\PY@tc##1{\textcolor[rgb]{0.73,0.13,0.13}{##1}}}
\expandafter\def\csname PY@tok@sd\endcsname{\let\PY@it=\textit\def\PY@tc##1{\textcolor[rgb]{0.73,0.13,0.13}{##1}}}
\expandafter\def\csname PY@tok@si\endcsname{\let\PY@bf=\textbf\def\PY@tc##1{\textcolor[rgb]{0.73,0.40,0.53}{##1}}}
\expandafter\def\csname PY@tok@se\endcsname{\let\PY@bf=\textbf\def\PY@tc##1{\textcolor[rgb]{0.73,0.40,0.13}{##1}}}
\expandafter\def\csname PY@tok@sr\endcsname{\def\PY@tc##1{\textcolor[rgb]{0.73,0.40,0.53}{##1}}}
\expandafter\def\csname PY@tok@ss\endcsname{\def\PY@tc##1{\textcolor[rgb]{0.10,0.09,0.49}{##1}}}
\expandafter\def\csname PY@tok@sx\endcsname{\def\PY@tc##1{\textcolor[rgb]{0.00,0.50,0.00}{##1}}}
\expandafter\def\csname PY@tok@m\endcsname{\def\PY@tc##1{\textcolor[rgb]{0.40,0.40,0.40}{##1}}}
\expandafter\def\csname PY@tok@gh\endcsname{\let\PY@bf=\textbf\def\PY@tc##1{\textcolor[rgb]{0.00,0.00,0.50}{##1}}}
\expandafter\def\csname PY@tok@gu\endcsname{\let\PY@bf=\textbf\def\PY@tc##1{\textcolor[rgb]{0.50,0.00,0.50}{##1}}}
\expandafter\def\csname PY@tok@gd\endcsname{\def\PY@tc##1{\textcolor[rgb]{0.63,0.00,0.00}{##1}}}
\expandafter\def\csname PY@tok@gi\endcsname{\def\PY@tc##1{\textcolor[rgb]{0.00,0.63,0.00}{##1}}}
\expandafter\def\csname PY@tok@gr\endcsname{\def\PY@tc##1{\textcolor[rgb]{1.00,0.00,0.00}{##1}}}
\expandafter\def\csname PY@tok@ge\endcsname{\let\PY@it=\textit}
\expandafter\def\csname PY@tok@gs\endcsname{\let\PY@bf=\textbf}
\expandafter\def\csname PY@tok@gp\endcsname{\let\PY@bf=\textbf\def\PY@tc##1{\textcolor[rgb]{0.00,0.00,0.50}{##1}}}
\expandafter\def\csname PY@tok@go\endcsname{\def\PY@tc##1{\textcolor[rgb]{0.53,0.53,0.53}{##1}}}
\expandafter\def\csname PY@tok@gt\endcsname{\def\PY@tc##1{\textcolor[rgb]{0.00,0.27,0.87}{##1}}}
\expandafter\def\csname PY@tok@err\endcsname{\def\PY@bc##1{\setlength{\fboxsep}{0pt}\fcolorbox[rgb]{1.00,0.00,0.00}{1,1,1}{\strut ##1}}}
\expandafter\def\csname PY@tok@kc\endcsname{\let\PY@bf=\textbf\def\PY@tc##1{\textcolor[rgb]{0.00,0.50,0.00}{##1}}}
\expandafter\def\csname PY@tok@kd\endcsname{\let\PY@bf=\textbf\def\PY@tc##1{\textcolor[rgb]{0.00,0.50,0.00}{##1}}}
\expandafter\def\csname PY@tok@kn\endcsname{\let\PY@bf=\textbf\def\PY@tc##1{\textcolor[rgb]{0.00,0.50,0.00}{##1}}}
\expandafter\def\csname PY@tok@kr\endcsname{\let\PY@bf=\textbf\def\PY@tc##1{\textcolor[rgb]{0.00,0.50,0.00}{##1}}}
\expandafter\def\csname PY@tok@bp\endcsname{\def\PY@tc##1{\textcolor[rgb]{0.00,0.50,0.00}{##1}}}
\expandafter\def\csname PY@tok@fm\endcsname{\def\PY@tc##1{\textcolor[rgb]{0.00,0.00,1.00}{##1}}}
\expandafter\def\csname PY@tok@vc\endcsname{\def\PY@tc##1{\textcolor[rgb]{0.10,0.09,0.49}{##1}}}
\expandafter\def\csname PY@tok@vg\endcsname{\def\PY@tc##1{\textcolor[rgb]{0.10,0.09,0.49}{##1}}}
\expandafter\def\csname PY@tok@vi\endcsname{\def\PY@tc##1{\textcolor[rgb]{0.10,0.09,0.49}{##1}}}
\expandafter\def\csname PY@tok@vm\endcsname{\def\PY@tc##1{\textcolor[rgb]{0.10,0.09,0.49}{##1}}}
\expandafter\def\csname PY@tok@sa\endcsname{\def\PY@tc##1{\textcolor[rgb]{0.73,0.13,0.13}{##1}}}
\expandafter\def\csname PY@tok@sb\endcsname{\def\PY@tc##1{\textcolor[rgb]{0.73,0.13,0.13}{##1}}}
\expandafter\def\csname PY@tok@sc\endcsname{\def\PY@tc##1{\textcolor[rgb]{0.73,0.13,0.13}{##1}}}
\expandafter\def\csname PY@tok@dl\endcsname{\def\PY@tc##1{\textcolor[rgb]{0.73,0.13,0.13}{##1}}}
\expandafter\def\csname PY@tok@s2\endcsname{\def\PY@tc##1{\textcolor[rgb]{0.73,0.13,0.13}{##1}}}
\expandafter\def\csname PY@tok@sh\endcsname{\def\PY@tc##1{\textcolor[rgb]{0.73,0.13,0.13}{##1}}}
\expandafter\def\csname PY@tok@s1\endcsname{\def\PY@tc##1{\textcolor[rgb]{0.73,0.13,0.13}{##1}}}
\expandafter\def\csname PY@tok@mb\endcsname{\def\PY@tc##1{\textcolor[rgb]{0.40,0.40,0.40}{##1}}}
\expandafter\def\csname PY@tok@mf\endcsname{\def\PY@tc##1{\textcolor[rgb]{0.40,0.40,0.40}{##1}}}
\expandafter\def\csname PY@tok@mh\endcsname{\def\PY@tc##1{\textcolor[rgb]{0.40,0.40,0.40}{##1}}}
\expandafter\def\csname PY@tok@mi\endcsname{\def\PY@tc##1{\textcolor[rgb]{0.40,0.40,0.40}{##1}}}
\expandafter\def\csname PY@tok@il\endcsname{\def\PY@tc##1{\textcolor[rgb]{0.40,0.40,0.40}{##1}}}
\expandafter\def\csname PY@tok@mo\endcsname{\def\PY@tc##1{\textcolor[rgb]{0.40,0.40,0.40}{##1}}}
\expandafter\def\csname PY@tok@ch\endcsname{\let\PY@it=\textit\def\PY@tc##1{\textcolor[rgb]{0.25,0.50,0.50}{##1}}}
\expandafter\def\csname PY@tok@cm\endcsname{\let\PY@it=\textit\def\PY@tc##1{\textcolor[rgb]{0.25,0.50,0.50}{##1}}}
\expandafter\def\csname PY@tok@cpf\endcsname{\let\PY@it=\textit\def\PY@tc##1{\textcolor[rgb]{0.25,0.50,0.50}{##1}}}
\expandafter\def\csname PY@tok@c1\endcsname{\let\PY@it=\textit\def\PY@tc##1{\textcolor[rgb]{0.25,0.50,0.50}{##1}}}
\expandafter\def\csname PY@tok@cs\endcsname{\let\PY@it=\textit\def\PY@tc##1{\textcolor[rgb]{0.25,0.50,0.50}{##1}}}

\def\PYZbs{\char`\\}
\def\PYZus{\char`\_}
\def\PYZob{\char`\{}
\def\PYZcb{\char`\}}
\def\PYZca{\char`\^}
\def\PYZam{\char`\&}
\def\PYZlt{\char`\<}
\def\PYZgt{\char`\>}
\def\PYZsh{\char`\#}
\def\PYZpc{\char`\%}
\def\PYZdl{\char`\$}
\def\PYZhy{\char`\-}
\def\PYZsq{\char`\'}
\def\PYZdq{\char`\"}
\def\PYZti{\char`\~}
% for compatibility with earlier versions
\def\PYZat{@}
\def\PYZlb{[}
\def\PYZrb{]}
\makeatother


    % Exact colors from NB
    \definecolor{incolor}{rgb}{0.0, 0.0, 0.5}
    \definecolor{outcolor}{rgb}{0.545, 0.0, 0.0}



    
    % Prevent overflowing lines due to hard-to-break entities
    \sloppy 
    % Setup hyperref package
    \hypersetup{
      breaklinks=true,  % so long urls are correctly broken across lines
      colorlinks=true,
      urlcolor=urlcolor,
      linkcolor=linkcolor,
      citecolor=citecolor,
      }
    % Slightly bigger margins than the latex defaults
    
    \geometry{verbose,tmargin=1in,bmargin=1in,lmargin=1in,rmargin=1in}
    
    

    \begin{document}
    
    
    \maketitle
    
    

    
    \hypertarget{basics-of-programming-i}{%
\section{Basics of Programming I}\label{basics-of-programming-i}}

    \hypertarget{introduction}{%
\subsection{Introduction}\label{introduction}}

    Throughout these workshops we are going to be teaching you about
programming using a language known as \textbf{Python}. Python is a
really powerful language, and it is very versatile. We are going to be
utilising an application known as \textbf{Jupyter Notebooks} which
allows us to create and share documents with code easily and in a
presentable visual format.

    Learning how to program is just like learning a real language. The only
way you can learn is by \emph{doing}, and it is really important to
understand that you will not learn anything unless you activley try and
get involved in the examples we present. You will not become an expert
in Python by the end of these workshops, but you will become more
literate, and in the same way that people who can speak Italian can
understand parts of Spanish these workshops will help your overall
ability to program and work with data in a coherent, efficient manner.

    \hypertarget{why-should-you-be-here-and-why-should-you-stay}{%
\subsection{Why should you be here? (and why should you
stay?)}\label{why-should-you-be-here-and-why-should-you-stay}}

    \begin{enumerate}
\def\labelenumi{\arabic{enumi}.}
\tightlist
\item
  \emph{Become more programming literate}: This is invaluable for your
  honours theses or any research project you may have to do
\item
  \emph{Your Resume}: Employers are becoming increasingly interested in
  graduates who are able to program or at least familiar with it.
\end{enumerate}

    \hypertarget{the-goal-of-this-program}{%
\subsection{The goal of this program}\label{the-goal-of-this-program}}

The goal of these workshops is to allow students to introduce students
to the fundamentals of programming. This is important because these
skills not only make students more employable, but are also essential
for any type of empirical work in Economics. This course endeavours to
be accessible to \emph{all students} regardless of background and/or
skill. Core to what we are trying to achieve is increase the level of
`programming literacy' amongst students. We must emphasise that this
program is \textbf{not compulsory} and we cannot be responsible for your
assignments or for anything related to the units you are enrolled in.

    \hypertarget{what-is-python-and-jupyter-notebooks}{%
\subsection{What is Python and Jupyter
Notebooks}\label{what-is-python-and-jupyter-notebooks}}

Python is a programming language, meaning that we can use it to make our
computer conduct calculations or manipulate data. Most of the tutorials
you will find on the internet might seem really confusing, a lot of them
will ask you to use the terminal or some editor such as Atom. That's
where Jupyter Notebooks comes in. Essentially, Jupyter Notebooks are a
program that allows us to write and run Python code in an envirionment
which is much more user friendly and which requires less background
knowledge.

\hypertarget{accessing-jupyter-notebooks-fill-in-either-ann}{%
\subsection{Accessing Jupyter Notebooks --- fill in, either
ann}\label{accessing-jupyter-notebooks-fill-in-either-ann}}

\hypertarget{working-directories}{%
\subsection{Working Directories}\label{working-directories}}

When we are working with any programming language, indeed any program
that requires us to read or save data we need to have some understanding
about what a `working directory' is. You can think of a working
directory as the location from which your program (in our case, Jupyter
Notebooks) operates from. If I want to read data for example, it either
needs to be in my working directory or I need to specify the path of the
file. We will see an example of this below. This can be a little
confusing at first, but for example my home directory is
\texttt{/Users/varunsatish}. Your working directory can be changed for
example to \texttt{/Users/varunsatish/Python\_Workshops}, however for
the moment this isn't too important it's just something you might need
to keep in mind for later on.

\hypertarget{packages}{%
\subsection{Packages}\label{packages}}

Packages are an important part of the Python programming language. You
can think of packages as `add ons' to the language. Some of these
packages are really the basis of a lot that you will do in the
programming language. A couple of examples are the Numpy and Pandas
packages, which we will see a little later. Generally, the syntax for
importing packages looks
like:\texttt{import\ {[}package{]}\ as\ {[}convenient\ name{]}}. For
example, \texttt{import\ pandas\ as\ pd} or
\texttt{import\ numpy\ as\ np}. If you look online there are names under
which packages are imported almost universally.

\hypertarget{troubleshooting}{%
\subsection{Troubleshooting}\label{troubleshooting}}

The two best troubleshooters anyone who needs to program has in their
arsenal are Google and Youtube. If you come across a problem, there has
almost definitely been someone who has made a tutorial about it, or a
forum post with a solved solution to the problem -- you just need to
understand what the question you are asking actually is. Part of this
comes with programming literacy, in all seriousness when it comes to
learning how to code, the best thing you can do is make sure you have
enough understanding not to write code automatically, but so that you
actually can ask the right questions when you inevitably stumble across
problems.

    \hypertarget{data-types-and-data-structures}{%
\subsection{1.1 Data Types and Data
Structures}\label{data-types-and-data-structures}}

    \hypertarget{what-is-a-variable}{%
\subsection{What is a variable?}\label{what-is-a-variable}}

In programming, a variable is an object that we use to write code.
Variables store information, and we can change what information these
variables hold. We can assign names to variables, which makes life a lot
easier when we are writing complicated code.

For example, all of you should be familar with variables from your
Econometrics classes. Suppose I observe an individual and I am
interested in their age, perhaps I want to do some calculations relating
to their age. Well, I can define a variable named \texttt{age}, and I
can assign some value to it.

In very very simple terms, a variable is an object we use to `remember
stuff', whether that be the output of some calculation, or simply a name
that is more convenient.

    \begin{figure}
\centering
\includegraphics{https://github.com/varunsatish/Coding-Tutorials/blob/master/images/variable_diagram.jpg?raw=true}
\caption{Variable}
\end{figure}

    Suppose I want to do some calculations based around the ages of a group
of people, I can make use of variables to make that much easier

    \begin{figure}
\centering
\includegraphics{https://github.com/varunsatish/Coding-Tutorials/blob/master/images/1_1_ppl_diagram.jpg?raw=true}
\caption{People}
\end{figure}

    \begin{Verbatim}[commandchars=\\\{\}]
{\color{incolor}In [{\color{incolor}3}]:} \PY{n}{pat} \PY{o}{=} \PY{l+m+mi}{21}
        \PY{n}{lea} \PY{o}{=} \PY{l+m+mi}{16}
        \PY{n}{tommy} \PY{o}{=} \PY{l+m+mi}{48}
        \PY{n}{varun} \PY{o}{=} \PY{l+m+mi}{22}
        \PY{n}{chris} \PY{o}{=} \PY{l+m+mi}{28}
\end{Verbatim}


    I can `run' a block of code by using: shift+enter. This will execute the
code contained within that block

    \hypertarget{some-common-errors}{%
\subsubsection{Some common errors:}\label{some-common-errors}}

\begin{enumerate}
\def\labelenumi{\arabic{enumi}.}
\tightlist
\item
  Using capital letters
\item
  Spaces
\item
  Numbers
\end{enumerate}

    \hypertarget{numbers-floats-and-integers}{%
\subsection{Numbers: Floats and
Integers}\label{numbers-floats-and-integers}}

The data type we have just been referring to are known as Integers.
These are one of the most fundamental data types in Python. Integers are
just whole numbers. In Python, as with in many other languages, we are
able to assign values to variables that we define, and we can perform
mathematical operations on or using them.

    \begin{Verbatim}[commandchars=\\\{\}]
{\color{incolor}In [{\color{incolor}5}]:} \PY{n}{total\PYZus{}age} \PY{o}{=} \PY{n}{pat} \PY{o}{+} \PY{n}{lea} \PY{o}{+} \PY{n}{tommy} \PY{o}{+} \PY{n}{varun} \PY{c+c1}{\PYZsh{}summing up ages}
        \PY{n+nb}{print}\PY{p}{(}\PY{n}{total\PYZus{}age}\PY{p}{)} \PY{c+c1}{\PYZsh{}printing ages}
\end{Verbatim}


    \begin{Verbatim}[commandchars=\\\{\}]
107

    \end{Verbatim}

    \textbf{\emph{NOTE:}} It is really important that you remember to
`comment' on your code. In Python, you can use \texttt{\#} before
something you type in order to write a section of code that does not
`run' when you run the code block. If you don't comment you will have
absolutely no idea what you were talking about when you go back and
check your code -- especially when your code gets complicated.

    As we can see, we don't need to keep defining new variables, once we
have defined a variable it has been `remembered'. Now, unlike other
programs that you may have used, for example STATA; there is no easy to
use GUI to track which variables have been created and so forth,
sometimes you can define a variable, delete the code, and if you haven't
restarted the kernel, the computer will still remember it.

    \hypertarget{words-strings}{%
\subsection{Words: Strings}\label{words-strings}}

Strings are an object that are \emph{generally} used to represent text
information. In the same way that we can assign an integer to a variable
\texttt{x}, we can do the same with strings. When we are working with
strings, it is important to remember to use the quotation marks
\texttt{\textquotesingle{}} or \texttt{"}. I can create collections of
say names using strings, for example I may interested in the names of
some students in a class.

    \begin{Verbatim}[commandchars=\\\{\}]
{\color{incolor}In [{\color{incolor}6}]:} \PY{n}{ECMT\PYZus{}3160} \PY{o}{=} \PY{p}{[}\PY{l+s+s1}{\PYZsq{}}\PY{l+s+s1}{Tommy}\PY{l+s+s1}{\PYZsq{}}\PY{p}{,} \PY{l+s+s1}{\PYZsq{}}\PY{l+s+s1}{Pat}\PY{l+s+s1}{\PYZsq{}}\PY{p}{,} \PY{l+s+s1}{\PYZsq{}}\PY{l+s+s1}{Lea}\PY{l+s+s1}{\PYZsq{}}\PY{p}{,} \PY{l+s+s1}{\PYZsq{}}\PY{l+s+s1}{Varun}\PY{l+s+s1}{\PYZsq{}}\PY{p}{]}
        \PY{n+nb}{print}\PY{p}{(}\PY{n}{ECMT\PYZus{}3160}\PY{p}{)}
\end{Verbatim}


    \begin{Verbatim}[commandchars=\\\{\}]
['Tommy', 'Pat', 'Lea', 'Varun']

    \end{Verbatim}

    Strings are \textbf{NOT} treated the same way as integers. For example,
\texttt{apple\ =\ 5} is treated differently to
\texttt{apple\ =\ \textquotesingle{}5\textquotesingle{}}, we
\textbf{cannot perform} numerical calculations using strings.

    \begin{Verbatim}[commandchars=\\\{\}]
{\color{incolor}In [{\color{incolor}19}]:} \PY{n}{z} \PY{o}{=} \PY{n}{x}\PY{o}{/}\PY{n}{y}
         
         \PY{n+nb}{type}\PY{p}{(}\PY{n}{z}\PY{p}{)}
\end{Verbatim}


\begin{Verbatim}[commandchars=\\\{\}]
{\color{outcolor}Out[{\color{outcolor}19}]:} float
\end{Verbatim}
            
    Essentially any time we are using text data, we are working with
strings. If there are problems with computation, sometimes the first
place you can look is by checking the \texttt{type()} of the data. For
example, if \texttt{type(z)\ =\ str} we cannot do any computation with
\texttt{z}.

    Notice that what we have actually done is created a collection of
objects. These collections are known as `lists' and they are very
important because they allow us to keep variables together. Python
doesn't (usually) have any notion of a `vector', so for those of you who
have some background in programming or Linear Algebra, sometimes when
you need to do some computation, you can think of lists of integers as
sort of like vectors.

    \hypertarget{collections-of-objects-lists}{%
\subsection{Collections of objects:
Lists}\label{collections-of-objects-lists}}

    \begin{figure}
\centering
\includegraphics{https://github.com/varunsatish/Coding-Tutorials/blob/master/images/lists.jpg?raw=true}
\caption{People}
\end{figure}

    A `list' is a collection of arbitrary objects, we define a list using
the square brackets:

    \begin{Verbatim}[commandchars=\\\{\}]
{\color{incolor}In [{\color{incolor}9}]:} \PY{n}{ECMT\PYZus{}3160} \PY{o}{=} \PY{p}{[}\PY{l+s+s1}{\PYZsq{}}\PY{l+s+s1}{Tommy}\PY{l+s+s1}{\PYZsq{}}\PY{p}{,} \PY{l+s+s1}{\PYZsq{}}\PY{l+s+s1}{Pat}\PY{l+s+s1}{\PYZsq{}}\PY{p}{,} \PY{l+s+s1}{\PYZsq{}}\PY{l+s+s1}{Lea}\PY{l+s+s1}{\PYZsq{}}\PY{p}{,} \PY{l+s+s1}{\PYZsq{}}\PY{l+s+s1}{Varun}\PY{l+s+s1}{\PYZsq{}}\PY{p}{]} \PY{c+c1}{\PYZsh{}usually I wouldn\PYZsq{}t have to redefine this variable, I just have for clarity}
\end{Verbatim}


    We can perform functions on lists, for example we may be interested in
how many elements are within the list, or we may be interested in a
specfic object within the list. These are very basic examples and there
are far more functions, but for now these will do.

    \begin{Verbatim}[commandchars=\\\{\}]
{\color{incolor}In [{\color{incolor}8}]:} \PY{n+nb}{len}\PY{p}{(}\PY{n}{ECMT\PYZus{}3160}\PY{p}{)} 
\end{Verbatim}


\begin{Verbatim}[commandchars=\\\{\}]
{\color{outcolor}Out[{\color{outcolor}8}]:} 4
\end{Verbatim}
            
    We can also \textbf{`index'} lists, meaning that we can refer to
specific elements within them by reffering to the numerical order. This
is extremley important later on when we talk about functions and loops.

    \begin{Verbatim}[commandchars=\\\{\}]
{\color{incolor}In [{\color{incolor}10}]:} \PY{n}{ECMT\PYZus{}3160}\PY{p}{[}\PY{l+m+mi}{0}\PY{p}{]} \PY{c+c1}{\PYZsh{}Python indexing begins at 0!}
\end{Verbatim}


\begin{Verbatim}[commandchars=\\\{\}]
{\color{outcolor}Out[{\color{outcolor}10}]:} 'Tommy'
\end{Verbatim}
            
    We have all come across indexing before. For example, if we have a
random variable \(x\), we can refer to the \(i^{th}\) observation with
\(x_i\) i.e the first observation is \(x_1\), the second is \(x_2\) and
so forth.

    \begin{Verbatim}[commandchars=\\\{\}]
{\color{incolor}In [{\color{incolor} }]:} \PY{c+c1}{\PYZsh{}Exercise 1.1 solution}
        
        \PY{n}{x} \PY{o}{=} \PY{l+m+mi}{1}
        
        \PY{n}{y} \PY{o}{=} \PY{l+s+s1}{\PYZsq{}}\PY{l+s+s1}{1}\PY{l+s+s1}{\PYZsq{}}
        
        \PY{n+nb}{print}\PY{p}{(}\PY{n}{x}\PY{p}{)}
         
        \PY{n+nb}{print}\PY{p}{(}\PY{n}{y}\PY{p}{)}
        
        \PY{n}{x} \PY{o}{+} \PY{l+m+mi}{3} 
        \PY{n}{y} \PY{o}{+} \PY{l+m+mi}{2}
\end{Verbatim}



    % Add a bibliography block to the postdoc
    
    
    
    \end{document}
